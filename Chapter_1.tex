\documentclass{article}
\usepackage{taoes}

\begin{document}
\title{Solutions for Chapter 1}

    \ex{1.30}
    $V_\out$ is the voltage at the output of the impedance voltage divider. We know that $Z_R = R$ and $Z_C = \frac{1}{j\omega C}$. So we have 
    \[V_\out = \frac{Z_C}{Z_R + Z_C} V_\in = \frac{\frac{1}{j \omega C}}{R + \frac{1}{j \omega C}} V_\in = \frac{1}{1 + j \omega R C} V_\in\]
    The magnitude of this expression follows:
    \[|V_\out| = \sqrt{V_\out V_\out^*} = \frac{1}{\sqrt{1 + \omega^2R^2C^2}}V_\in\]
    where $V_\out^*$ is the complex conjugate of $V_\out$
    
    \ex{1.37}
    The Norton equivalent circuit can be found by measuring $I_\short$ and $V_\open$:
    
    \[I_\Norton = I_\short = \frac{10\V}{10\k\Omega} = \sol{1\m\A}\] 
    
    and since $V_\open = 5\V$, we have
    
    \[R_\Norton = \frac{5\V}{1\m\A} = \sol{5\k\Omega}\]
    
\end{document}
