\documentclass{article}
\usepackage{taoes}

\begin{document}
\title{Solutions for Chapter 2}

	\ex{2.1}
	We assume a forward voltage for the LED of 1.5\V. Then for $I_{LED}$ we have
	\[I_{LED} = \frac{V_R}{R} = \frac{3.3\V - 1.5\V}{330\Ohm} \approx \sol{5.5\m\A}\]
	To estimate the $\beta_{min}$ we need the current entering the base
	\[I_B = \frac{3.3\V - 0.6\V}{10\k\Ohm} = 0.27\m\A \]
	Thus
	\[\beta_{min} \geq \frac{I_{LED}}{I_B} = \sol{20}\]
	
	\ex{2.2}
	
	\tans{NOTE: According to the errata $0.63$ should be replaced by $0.76$ and $63\u\sec$ by $76\u\sec$.}
	\\\\
	Starting from the hint that the capacitor charges from $-4.4\V$ towards $+5\V$, we would result to a total $9.4\V$ for a full charge. However, the $V_{BE}$ of $Q_2$ is clipping the charging process at only $5\V$ of the total (from $-4.4\V$ to $0.6\V$). Thus, the capacitor will be $53\%$ charged at the end. \\Solving the voltage equation for a charging capacitor gives us
	\[V_C(t) = V_f * (1 - e^{-\frac{t}{R C}})\]
	set $V_C(t_1) = 0.53 * V_f$
	\[0.53 = 1 - e^{-\frac{t_1}{R_3 C_1}}\]
	\[\Rightarrow\]
	\[t_1 = - R C*ln(0.47) \approx \sol{0.76 * R_3 C_1}\]
	
	\ex{2.3}
	
	The output voltage is reduced due to the $R_4 - R_5$ voltage divider
	\[V_\out = \frac{R_5}{R_4 + R_5} * (V_{CC} - 0.6\V) \approx \sol{4.18\V}\]
	To estimate the minimum $\beta_3$, we need first to find the maximum (worst-case) collector current for which $Q_3$ should still be in saturation. For this we can assume a $0\V$ drop across C and $Q_3$ while the current travels through the parallel connected resistors $R_2||R_3$.
	\[I_{C_3,max} = \frac{V_{CC}}{R_2||R_3} = 5.5\m\A\]
	\[\Rightarrow \beta_{3,min} = \frac{I_{C_3,max}}{I_{B_3}} = \frac{5.5\m\A}{\frac{(4.18\V - 0.6\V)} {20\k\Ohm}} \approx \sol{31}\]
\end{document}
