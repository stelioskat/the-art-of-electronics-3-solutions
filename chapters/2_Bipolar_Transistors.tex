\documentclass[../main.tex]{subfiles}

\begin{document}
\section{Bipolar Transistors}

\ex{2.1}
	We assume a forward voltage for the LED of 1.5\V. Then for $I_{LED}$ we have
	\[I_{LED} = \frac{V_R}{R} = \frac{3.3\V - 1.5\V}{330\Ohm} \approx \sol{5.5\m\A}\]
	To estimate the $\beta_{min}$ we need the current entering the base
	\[I_B = \frac{3.3\V - 0.6\V}{10\k\Ohm} = 0.27\m\A \]
	Thus
	\[\beta_{min} \geq \frac{I_{LED}}{I_B} = \sol{20}\]
	
\ex{2.2}
	
	\tans{NOTE: According to the errata $0.63$ should be replaced by $0.76$ and $63\u\sec$ by $76\u\sec$.}
	\\\\
	Starting from the hint that the capacitor charges from $-4.4\V$ towards $+5\V$, we would result to a total $9.4\V$ for a full charge. However, the $V_{BE}$ of $Q_2$ is clipping the charging process at only $5\V$ of the total (from $-4.4\V$ to $0.6\V$). Thus, the capacitor will be $53\%$ charged at the end. \\Solving the voltage equation for a charging capacitor gives us
	\[V_C(t) = V_f * (1 - e^{-\frac{t}{R C}})\]
	set $V_C(t_1) = 0.53 * V_f$
	\[0.53 = 1 - e^{-\frac{t_1}{R_3 C_1}}\]
	\[\Rightarrow\]
	\[t_1 = - R_3 C_1*ln(0.47) \approx \sol{0.76 * R_3 C_1}\]
	
\ex{2.3}
	
	The output voltage is reduced due to the $R_4 - R_5$ voltage divider
	\[V_\out = \frac{R_5}{R_4 + R_5} * (V_{CC} - 0.6\V) \approx \sol{4.18\V}\]
	To estimate the minimum $\beta_3$, we need first to find the maximum (worst-case) collector current for which $Q_3$ should still be in saturation. For this we can assume a $0\V$ drop across C and $Q_3$ while the current travels through the parallel connected resistors $R_2||R_3$.
	\[I_{C_3,max} = \frac{V_{CC}}{R_2||R_3} = 5.5\m\A\]
	\[\Rightarrow \beta_{3,min} = \frac{I_{C_3,max}}{I_{B_3}} = \frac{5.5\m\A}{\frac{(4.18\V - 0.6\V)} {20\k\Ohm}} \approx \sol{31}\]
	
\ex{2.4}
	
	By using KCL and the fact that the transistor is in the active region we get
	\[i_E = i_C + i_B = (\beta + 1) * i_B = (\beta + 1) \frac{v_B}{Z_{source}}\]
	For small signals $Z_\out = \frac{v_E}{i_E} = \frac{v_B}{i_E}$. Thus:
	\[ \sol{Z_\out = \frac{v_B}{(\beta + 1)\frac{v_B}{Z_{source}}} = \frac{Z_{source}}{\beta +1}}\qquad q.e.d.\]
	Note: In practice one will often see $Z_\out \approx \frac{Z_{source}}{\beta}$. When $\beta \approx 100$ the "$+ 1$" part is often being ignored to simplify calculations.

\ex{2.5}
	
	\begin{schematic}{fig:2.5.1}{Follower driven by voltage divider}
		(0,-1) node[ground](GND1){}
		(GND1) to[R=$R_2$] (0,2)
		to[R=$R_1$] (0,4)
		(0,4) node[vcc] {$+15\V$}
		
		(3,2) node[npn] (Q1) {Q}
		(0,2) to[short] (Q1.B)
		(3,2) node[npn] (Q1) {Q}
		(3,4) node[vcc] {$+15\V$}
		(3,4) to[short] (Q1.C)
		(3,-1) node[ground](GND2){}
		(Q1.E) to[R=$R_E$] (GND2)
		(Q1.E) to[short, -o] ++(2,0) node[right]{$+5\V$}
		(Q1.B) node[above]{$5.6\V$}
	\end{schematic}
	
	We can simplify the voltage divider with it equivalent Th\'evenin voltage source depicted in Figure~\ref{fig:2.5.2} below.
	
	\begin{schematic}{fig:2.5.2}{Follower driven by equivalent Th\'evenin source}
		(0,-1) node[ground](GND1){}
		(GND1) to[V, l=$V_\Th$, a={$+5.6\V$}, invert] (0,2) 
       
		(3,2) node[npn] (Q1) {Q}
		(0,2) to[R=$R_\Th$] (Q1.B)
		(3,4) node[vcc] {$+15\V$}
		(3,4) to[short] (Q1.C)
		(3,-1) node[ground](GND2){}
		(Q1.E) to[R=$R_E$] (GND2)
		(Q1.E) to[short, -o] ++(2,0) node[right]{$+5\V$}
	\end{schematic}
	
	With $R_\Th = R_1||R_2$ which is also our $R_{source}$. The output impedance of our circuit then is  
	\[R_\out=\frac{R_\Th}{\beta + 1} \approx \frac{R_\Th}{100} = \frac{R_1||R_2}{100}\qquad(\text{assuming}~\beta \approx 100)\]
	 
	We also know that the following condition needs to be true in order to achieve the wished $5.6\V$ at the base: 
	\[\frac{R_2}{R_1 + R_2} = \frac{5.6\V}{15\V} \Rightarrow R_2 \approx 0.6 * R_1\]
	
	Now let's observe the equivalent circuit \emph{with} load.
	
	\begin{schematic}{fig:2.5.3}{Equivalent voltage source of emitter follower with load}
		(3,-2) node[ground](GND1){}
		(0,0) node[left]{$+5\V$}
		to[R=$R_\out$, o-] (3,0)
		to[short, -o] (4,0) node[right]{$V\out$}
		(3,0) to[R=$R_L$] (GND1)
	\end{schematic}
	
	Our goal is to have a maximum voltage drop of 5\% with maximum load:
	
	\[V_\out = \frac{R_L}{R_\out + R_L} * 5\V \geq 0.95 * 5\V\]
	$\Rightarrow$
	\[R_\out \leq \frac{R_L}{19} \leq \frac{\frac{4.75\V}{25\m\A}}{19} = 10\Ohm\]
	$\Rightarrow$
	\[\frac{R_\Th}{100} \leq 10\Ohm\]
	$\Rightarrow$
	\[\frac{R_1||R_2}{100} \leq 10\Ohm\]
	$\Rightarrow$
	\[R_1 \leq 2.7\k\Ohm~\text{and}~R_2 \leq 1.62\k\Ohm\]
	
	We choose the following values for $R_1$ and $R_2$:
	\[\sol{R_1 = 2.7\k\Ohm}~\text{and}~\sol{R_2 = 1.6\k\Ohm}\]
	
	NOTE 1: We could have picked more conservative (smaller) values for the resistors. However, this would increase the idle power consumption.\\\\
	NOTE 2: We could also define a value for $R_E$ i.e. to limit the quiescent current, but this is out of the scope of this exercise. Basically we see now $R_E$ as our load or put differently, our total load is $R_E||Z_{whatever-the-user-wants}$.
	
\end{document}
